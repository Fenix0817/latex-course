\documentclass[12pt]{article}
\usepackage[brazil]{babel} 
\usepackage[T1]{fontenc} % Adicionado para poder usar os Caracteres Portugueses
\usepackage[utf8]{inputenc}
\usepackage{url}
\usepackage[style = numeric,  
sortcites,language = brazil]{biblatex}
\usepackage{csquotes}
\addbibresource{bib-exercise.bib}


\title{Dez Segredos Para uma Boa Apresentação}
\author{Você}

\begin{document}
\maketitle

\section{Introdução}

O Texto constante neste exercício é uma versão traduzida e resumida do excelente artigo da autoria de Mark Schoeberl e Brian Toon:
\url{http://www.cgd.ucar.edu/cms/agu/scientific_talk.html}

\section{Os Segredos}


Compilei esta lista pessoal de ``Segredos'' depois de ouvir apresentações eficazes e ineficazes de diversos oradores. Esta lista não é compreensiva --- De certeza que coisas ficaram de fora. Mas, esta lista contêm 90\% do que tu deves saber e deves fazer.


\begin{enumerate}

\item Prepara o teu material cuidadosamente e logicamente. Conta uma história.

\item Pratica o teu discurso. Falta de preparação não é uma desculpa.

\item Não tenhas demasiado material. Bons Oradores apresentam um ou dois pontos principais e focam-se neles.

\item Evita equações. Diz-se que por cada equação na tua apresentação, o número de pessoas que te compreendem passa para metade. Isto é, sendo $q$ o número de equações na apresentação, o número de pessoas que percebem a apresentação é dado por:

\begin{equation}
n = \gamma \left( \frac{1}{2} \right)^q
\end{equation}
em 	que $\gamma$ é a constante de proporcionalidade\cite{Valente1999,Boyce2010}.

\item A conclusão deve conter apenas os pontos chave. As pessoas não se lembram de mais que 2 ou 3 coisas de uma apresentação, principalmente se tiverem ouvido várias apresentações em grandes conferências.

\item Fala para audiência e não para a tela. Um dos erros mais comuns é o orador falar virado de costas para audiência. 

\item Evita fazer sons que distraiam audiência. Evita os ``Ummm'' ou ``Ahhh'' entre frases.

\item Melhora os teus gráficos. Uma pequena lista de dicas para melhorar gráficos numa apresentação:

\begin{itemize}
\item Usa uma fonte grande.

\item Mantém os gráficos simples. Não mostres gráficos que não precisas.

\item Usa cores.

\end{itemize}

\item Seja pessoal a responder a questões \cite{Strang2013,Saviani1980}.

\item Usa humor sempre que possível. É fascinante como uma piada seca consegue fazer as pessoas rir numa conferência científica.

\end{enumerate}

\printbibliography
\end{document}
